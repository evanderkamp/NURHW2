\section{Heating and cooling in HII regions}

Because I could copy all of the functions from the working classes, 
my code for those is similar to my sister\'s, Liz van der Kamp (s2135752). 
For this exercise I needed to use different root finding algorithms to find the roots of functions, and I 
used Newton Rhapson, Bisection, and Secant.

The code I wrote for this is:
\lstinputlisting{NUR_handin2Q2.py}

\section{a}

For a) I needed to find the root of the function $\Gamma_{pe} - \Lambda_{rr}$ to find the equilibrium 
temperature. $\alpha_B, n_h$ and $n_e$ cancel out because they appear in both expressions.
We want an accuracy of at least 0.1K, but the function is around zero for a long time (I plotted it to see) 
so we want a high accuracy.
Secant diverges, so I use Newton-Rhapson with $10^{3.5}$ as initial guess. It took 3 steps to find a temperature 
with a function value of ~$10^{-19}$. I measured the time it took to find the root and output it together with
the found T. The output is of the form [T (in K), time taken (in sec)]:
\lstinputlisting{NUR2Q2TandttNR.txt}

To make sure I was actually near a root and not just in a sea of small function values, I plotted the 
equilibrium function, a horizontal line at 0, and the T value found by my NR root-finder. I set the 
x-limits to 0.1K around the T value found by my root-finder to make sure I was close enough.
The result you can see in Fig. \ref{fig:fig1}.

\begin{figure}[h!]
  \centering
  \includegraphics[width=0.9\linewidth]{NUR2Q2plot1.pdf}
  \caption{Plot corresponding to exercise 2a, showing the equilibrium function, the root found by NR, and
  a horizontal line at 0.}
  \label{fig:fig1}
\end{figure} 
