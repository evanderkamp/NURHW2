\section{Satellite galaxies around a massive central}

For this exercise, I use functions for trapezoid integration, Romberg integration, a combination of 
LCG and 64bit XOR random number generator, Quicksort, and Ridder differentiation I made during tutorials in 
which I worked together with Liz van der Kamp (s2135752).


The code I wrote is:
\lstinputlisting{NUR_handin2Q1.py}

\subsection*{a}

For a) we have that x = $r/r_vir$, so integrating over spherical coordinates gives dx = $dr/r_vir$, 
and n(x) doesnt depend on theta or phi so the 3D integral is simply $4\pi$ times the integral over 
$r^2$*dr (or $r_vir^3*x^2*dx$)
To solve for A we can rewrite the integral to get $(4\pi/(A*N_sat))*\int n(x)*x^2 dx = 1/(A*r_vir^3)$
Because Romberg integration gives the most accurate result, I used Romberg integration with 50 steps 
(so a step size of 5/50) and an order of 6 to get 1/A (taking the $r_vir$ into the constant A),
I then inverted the output of the integration to get A:
\lstinputlisting{NUR2Q1Asol.txt}

\subsection*{b}

For b) we have that N(x)dx is $4\pi*n(x)*x^2 dx$, because integrating that divided by Nsat gives 1, 
so integrating p(x)dx = N(x)dx/Nsat also gives 1, which is what we want for a probability distribution.
To sample this distribution, I use rejection sampling for 10000 x values between 0 and 5,
where I use a combination between 64bit XOR and LCG RNG to generate a random x and y and accept or reject
based on whether $y \leq p(x)$. For every iteration I use a different seed in the RNG to make sure that I 
don\'t constantly get the same answer for x and y, and because I want to generate a number between 0 and 1 
and I divide by the maximum of the generated numbers in the RNG, I cannot request just 1 number 
(then it will always be 1), so I generate 50 and take the first and second number for x and y.

The plot this produces can be seen in Fig \ref{fig:fig11}. The bins below ~$5*10^{-3}$ are empty because there 
the probability reaches ~$10^{-4}$ so with 10000 points you expect at most 1 count there. 
I used density=True in the plt.hist function because that does the normalization by dividing the bins by 
their width and the total number of counts.

\begin{figure}[h!]
  \centering
  \includegraphics[width=0.9\linewidth]{NUR2Q1plot1.pdf}
  \caption{Plot corresponding to exercise 1b, showing a log-log plot of p(x) and the histogram of 10000 
  sampled values in 20 logarithmically-spaced bins between x = $10^{-4}$ and x = 5.}
  \label{fig:fig11}
\end{figure} 


\subsection*{c}

For c) to select 100 random satellites, I want to randomly shuffle the sample array I produced in b) and then for 
example take the first 100 galaxies from the shuffled array. This ensures to select every galaxy with equal
probability, not drawing the same twice, and not rejecting any draw. 
To do this, I generate an array of the same length as my sample (10000) filled with random numbers (so using my RNG)
then, I sort the random array with my Quicksort algorithm, and get back the indices that were switched to sort it.
With that array of indices, I shuffle the sample, and I take the first 100 galaxies as my random sample.

Then, I sort that random sample with my Quicksort algorithm in order to plot the number of galaxies within a radius.
The sorted array allows me to easily see how many galaxies are within that radius. 
Namely, the first element is the smallest radius within which there is 1 galaxy, the second element gives the radius 
within which there are 2 galaxies, and so on. 
So, we get an x array from x = $10^{-4}$ to x = 5, which is simply the random sample with 
x = $10^{-4}$ and x = 5 inserted at the beginning and end, and a counts array which has 0 (at x = $10^{-4}$) 
until 100 with another 100 appended at the end (for x = 5). 
The resulting plot can be seen in Fig. \ref{fig:fig12}.


\begin{figure}[h!]
  \centering
  \includegraphics[width=0.9\linewidth]{NUR2Q1plot2.pdf}
  \caption{Plot corresponding to exercise 1c, showing a xlog between x = $10^{-4}$ and x = 5 showing the 
number of galaxies within a certain radius.}
  \label{fig:fig12}
\end{figure} 


\subsection{d}

For d) I use Ridder differentiation because that yields the most accurate result. I used a stepsize of 0.1
and order 2 to get the differential at x=1 up to an accuracy below $10^{-10}$. To see if that accuracy has been
reached, I compare to the analytical derivative of n(x), which is 
$Nsat* A*((x)^{a-4}/(b)^{a-3})*exp(-(x/b)^c)*(a-3-c*(x/b)^c)$.
I reach an accuracy of 8*$10^{-15}$ within 4 steps for the Ridder differentiation.
The analytical derivative and the calculated derivative with Ridder are outputted as:
\lstinputlisting{NUR2Q1derivs.txt}


